%!TeX root=../tese.tex
%("dica" para o editor de texto: este arquivo é parte de um documento maior)
% para saber mais: https://tex.stackexchange.com/q/78101/183146

% Insira aqui os metadados do seu trabalho. Para isso, copie,
% com as alterações necessárias, o conteúdo do arquivo
% conteudo-exemplo/metadados.tex

\title{
    % Obrigatório nas duas línguas
    titlept={<Empresas DNA USP>},
    titleen={<Empresas DNA USP>},
    % Opcional, mas se houver deve existir nas duas línguas
    subtitlept={<Subtítulo>},
    subtitleen={<Subtitle>},
    % Opcional, para o cabeçalho das páginas
    shorttitle={Título curto},
}
\author{Lucas Toshio Loschner Fujiwara}
\date


\orientador{Prof. Alfredo Goldman}
\coorientador[fem]{Profª. Geciane Silveira Porto}

\resumo{
Elemento obrigatório, constituído de uma sequência de frases concisas e
objetivas, em forma de texto.  Deve apresentar os objetivos, métodos empregados,
resultados e conclusões.  O resumo deve ser redigido em parágrafo único, conter
no máximo 500 palavras e ser seguido dos termos representativos do conteúdo do
trabalho (palavras-chave). Deve ser precedido da referência do documento.
Texto texto texto texto texto texto texto texto texto texto texto texto texto
texto texto texto texto texto texto texto texto texto texto texto texto texto
texto texto texto texto texto texto texto texto texto texto texto texto texto
texto texto texto texto texto texto texto texto texto texto texto texto texto
texto texto texto texto texto texto texto texto texto texto texto texto texto
texto texto texto texto texto texto texto texto.
Texto texto texto texto texto texto texto texto texto texto texto texto texto
texto texto texto texto texto texto texto texto texto texto texto texto texto
texto texto texto texto texto texto texto texto texto texto texto texto texto
texto texto texto texto texto texto texto texto texto texto texto texto texto
texto texto.
}

\abstract{
Elemento obrigatório, elaborado com as mesmas características do resumo em
língua portuguesa. De acordo com o Regimento da Pós-Graduação da USP (Artigo
99), deve ser redigido em inglês para fins de divulgação. É uma boa ideia usar
o sítio \url{www.grammarly.com} na preparação de textos em inglês.
Text text text text text text text text text text text text text text text text
text text text text text text text text text text text text text text text text
text text text text text text text text text text text text text text text text
text text text text text text text text text text text text.
Text text text text text text text text text text text text text text text text
text text text text text text text text text text text text text text text text
text text text.
}
